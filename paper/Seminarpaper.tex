% !TeX spellcheck = en_US
% This is LLNCS.DOC the documentation file of
% the LaTeX2e class from Springer-Verlag
% for Lecture Notes in Computer Science, version 2.4
\documentclass{llncs}
\usepackage{llncsdoc}
\usepackage{graphicx} 
%
\begin{document}
\thispagestyle{empty}
\rule{\textwidth}{1pt}
\vspace{2pt}
\begin{flushright}
\Huge
\begin{tabular}{@{}l}
Barriers to the\\
implementation of\\
k-anonymity and\\
related microdata\\
anonymization techniques\\
in a realworld application\\[6pt]

\end{tabular}
\end{flushright}
\rule{\textwidth}{1pt}
\vfill
\title{Barriers to the implementation of k-anonymity and related microdata anonymization techniques in a realworld application}
\author{Andreas Wiegnand, 1878334\\
	Ludwig Schallner, 1850413}
\institute{}
\maketitle
%
%\tableofcontents
\newpage
\setcounter{page}{1}
\section{Introduction}
%
Nowadays data is a key factor in nearly every domain. It is comparable to the gold rush of the 19. century \cite{datarevo}. Furthermore, storage space and network connectivity become affordable \cite{sweeney2002k}. But to use the data for commercial or scientific purposes the privacy of the data holder does not have to be compromised or in other words the data holder need to know how to produce anonymous data otherwise the database cannot survive, because if the information of the table gets released there will be no need anymore for this data \cite{sweeney2002k}.\\

The papers The so-called k-anonymity method, which produces anonymous data, theoretical. But there are practical barriers that will occur in the real world. Those prevent such implementation. The goal of k-anonymity is, to prevent the possibility to get information about the real individual, or at least with k other possible individuals. So if a individual is described by a tuple of \ensuremath{\langle f_1, ... ,f_n \rangle} features and each feature can have \ensuremath{\langle a_1,...,a_n \rangle} attributes. There are at least k another individual with the same attribute for each feature so that there is no possibility to reduce the real individual and there will be at least k individuals with the same tuple\cite{sweeney2002k}.\\

The attributes that are used to link the external data is called quasi-identifiers. Typical values for them are gender, date of birth and zip code \cite{ldiversity}. 
We will present techniques that override k-anonymity and get the real individual. Another problem we will introduce is, that the producing of k-anonymity of a computational view is an NP-hard problem, like Meyersond and Williams shown.
...

\newpage
\section{Basics}


\textbf{Microdata:}\\
First of all, it should be clear what microdata is, those data is containing records of information about individuals. The upside versus the more known summary or aggregate data is, that microdata is naturally flexible. Everyone who has this data can perform own statistics from that data \cite{microdataweb}.\\
\textbf{Identifier:}\\
Indentifier Definition\\
\textbf{Quasi-identifier:}\\
Even though explicit identifier got removed from published data. Such an explicit attribute, which would not uniquely identify the record owner. But if combined (with other non-explicit attributes), they become explicit identifier. Which resulting that those can link towards the owner. In such a case those attributes are called quasi-identifier \cite{dalenius1986finding}. Such process is shown in figure \ref{quasiidentifier}.\\
\textbf{Sensitive data:}\\
Sensitive Data Definition\\
\textbf{Brackground knowlegde:}\\
Defintion\\
\textbf{K-Anonymity}




\section{Implementing of k-anonymity}

Like Dalenius already mentioned it is absolutely necessary that an attacker, under no circumstances, can learn about whatsoever target if he is studying the published database. Not even if the attacker has background knowledge from any other sources  \cite{Dalenius1977}. Unfortunately like Dwork showed 2006 that such safety is impossible because of background knowledge. For example, if the attacker knows that Bob get paid twice as the average German man and the attacker got access to a database which publishes the average income by German men. The anonymity of Bob is compromised even if Bob's data is not in the database \cite{dwork2011differential}.  

\paragraph{GRAMMAR CHECKED VIA grammarly.com ENDE VON NEUEN ZEUG}

\subsection{Linking data}
A barrier to do the implementation of k-anonymity, the attacker can take another dataset and link both together to get rid off the k-anonymity and infer the real individual. This process is called linking data and was first described by Sweeney\cite{sweeney2002k}. She showed that with a example of health care data from 37 states in the USA. The institute from which she bought the data, insures the anonymity of the individuals. Sweeney purchased the voter registration list for Cambridge Massachusettts and received information of the voters including ZIP code, birth date and gender (non explicit identifier) of each voter. She linked that information with the medical data. It was possible to deanonymize the data  and get ethnicity, visit date, diagnosis, procedure, medication and total charge of some patients \cite{sweeney2002k}. 

\begin{figure}[]
	\centering
	\includegraphics[width=0.7\textwidth]{linkingdata.png}
	\caption{linking data}%
	\label{quasiidentifier}
\end{figure}
You got two datasets A and B. Each dataset got \ensuremath{\langle f_1, ... ,f_n \rangle} features and \ensuremath{\langle r_1, ... ,r_n \rangle} rows.
Each row is then a tuple \ensuremath{r_i} with n features \ensuremath{\langle f_1, ... ,f_n \rangle} describing the individual.
Even tho the data is k-anonimized you can get rid oft he anonymity of the individual by linking the A to B. So if \ensuremath{A \cap B \not \neq \emptyset} it is possible to infer the anonymized individual \cite{sweeney2002k}.
As a result any attacker who know such data (ZIP Code, Birth date and sex) could easily identify with such an attack his victim. For example Peter see his ex-wife at the doctor, most likely he knows her ZIP-Code, Birth date and sex. Therefore he finds out what she is suffer from. 
\subsection{Unsorted matching attack against k-anonymity}
There is a possibility of a leak of information, if the release k-anonymity data is in some kind of a sort release. This mean the numerical attributes are descending or ascending sorted and attributes, which be of characters are alphabetical ordered, can give the attacker Information about the sensitive data. To prevent this attack, just get the data into a random order with a pseudo randomized sorting algorithm \cite{sweeney2002k}. As an example take a look at the table 3: matching attack  will give an example on that. If you compare the different release generalized tables you can figure out all quasi identifier of those \cite{sweeney2002k}.
\\...
\begin{table}	
	\caption{matching attack}
	\centering
	\begin{tabular}[t]{|l|l|}		
	\hline
	Age & ZIP   \\ \hline
	2   & 91058 \\ 
	4   & 91058 \\ 
	50  & 27785 \\ 
	52  & 27785 \\ 
	20  & 32105 \\ 
	21  & 32105 \\ 
	31  & 67676 \\ 
	32  & 67676 \\ \hline
		\end{tabular}
	\hfill
	\begin{tabular}[t]{|l|l|}
	\hline
	Age & ZIP   \\ \hline
	*   & 91058 \\
	**   & 91058 \\
	5*  & 27785 \\
	5*  & 27785 \\
	2*  & 32105 \\
	2*  & 32105 \\
	3*  & 67676 \\
	3*  & 67676 \\ \hline  
	\end{tabular}
	\hfill
	\begin{tabular}[t]{|l|l|}
	\hline
	Age & ZIP   \\	\hline
	2   & 91*   \\
	4   & 91*   \\
	50  & 27*   \\
	52  & 27*   \\
	20  & 32*   \\
	21  & 32*   \\
	31  & 67*   \\
	32  & 67*   \\ \hline  
	\end{tabular}
\end{table}
\subsection{Complementary release attack against k-anonymity}

The problem of complementary release attack against k-anonymitym lies by the release of other k-anoymitizet tables from the same dataset. To stop the attack the data holder should consider for each release of data if it's possible to release information with older released data. This is hard to avoid especially when the data can come from different individuals \cite{sweeney2002k}.
\\...
\subsection{Temporal attack against k-anonymity}

The data can be change over time. New tuples might be added or persistent ones can be changed. If the GT0 was release at time T=0 and on a later time GT1 will be released at time T = 1 with new tuples of information. Both tables at their time stamp T are k-anonymity, but it will not be checked if they are k-anonymity between them. So their is a possibility of information leaking and a failure oft he k-anonymity in both tables \cite{sweeney2002k}.
\\...
\subsection{Homogeneity Attack}


\subsection{Background Knowledge Attack}

Taking background knowledge attack from a person and take it into account to derive the sensitive data. In our example of table 1 this could be that we know one person my name, age, and nationality. Additionally we know, because that she is asian, in would be unusual that she got diabetes, because diabetes is a uncommon sickness in japan \cite{ldiversity}.
\\...
\subsection{Complexity of producing k-anonymity}
Till now we only looked at problems of information leaking and privacy problems for individuals. Data is personal-specific information which is structured as a table in rows and columns. Rows a tuple. The columns are attributes with are a set of values which describe the certain attribute. A tuple specify a person. K-anonymity is about protecting the identity of a person not relationships of companies or governments. So the goal of k-anonymity is, not getting more information by linking the data to external data. The bridge between the data and external data is called "quasi-identifier". Examples for that would be ZIP, gender, birth date etc.. \\
\\
Generalization mean, replacing a value with a less specific but semantic identical value. For example we got a list of forenames of buys, (Achmed, Achilles, Achim). To generalize this names you can just (Ach*,Ach*, Ach*) delete the last chars of the name. So there is a less specific domain and now more generalize through this mapping. Suppression on the other hand means not releasing the value at all.

\newpage
\bibliography{literature}
\bibliographystyle{splncs03}
\end{document}
